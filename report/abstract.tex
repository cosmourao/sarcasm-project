\thispagestyle{empty}
\vspace*{\fill}
\begin{center}
    \textbf{Abstract}
\end{center}
Converting text into numerical representations is a key part of Natural Language Processing (NLP). Vectorization methods can vary in complexity, and some capture linguistic features that others do not, such as contextual differences of words. Sarcasm identification in text is a highly context-dependent task where positive words can have negative connotations and \textit{vice-versa}. This report focused on the comparison of different word vectorization techniques, namely Term Frequency - Inverse Document Frequency (TF-IDF), pre-trained Continuous-Bag-of-Words (CBOW) and distilled Bidirectional Encoder Representations from Transformers (distilBERT) embeddings and their sarcasm classification accuracy using a Linear Support Vector Classifier (L-SVC) against a random choice baseline. The results show that sarcasm classification benefits from context-sensitive embeddings: The baseline, TF-IDF, CBOW reached, respectively, 50\%, 79\% and 81\% accuracy on test data. Meanwhile, distilBERT embeddings achieved 86\% accuracy. The analysis had limitations as embeddings were not fine-tuned for the task at hand.
\vspace*{\fill}